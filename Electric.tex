\chapter{电子电路}

\section{电源}

    \subsection{同步Buck}
    开关频率为$f_s$,则周期$T_s=\frac{1}{f_s}$,MTOP(同步Buck,上管称开关管,记MTOP,下管称续流管,记MBOT)导通时长为
    \begin{equation}
        T_\mathrm{on}=D\cdot\frac{1}{f_s}
    \end{equation}
    $D$为占空比。
    电感开始储能,流过的电流增大,电感上电压左正($V_\mathrm{in}$)右负($V_\mathrm{out}$),伏秒积:
    \begin{equation}
        (V_\mathrm{in}-V_\mathrm{out})DT_s
    \end{equation}
    (伏秒积反映了电感中电流的增量。这是因为电感特性:电感电压等于感量乘以电流的时间变化率。)
    其本质为流过的电流变化:
    \begin{equation}
        i_L(t)=i_L(0)+\frac{V_\mathrm{in}-V_\mathrm{out}}{L}t
    \end{equation}
    因此电流增量:
    \begin{equation}
        \Delta I_L=i_L(T_\mathrm{on})-i_L(0)=\frac{\left(V_\mathrm{in}-V_\mathrm{out}\right)D}{Lf_s}
    \end{equation}
    MTOP关断,MBOT导通时长为:
    \begin{equation}
        T_\mathrm{off}=(1-D)\cdot\frac{1}{f_s}
    \end{equation}
    电感向外输出电能,流过的电流逐渐减小,电感上电压左负(0)右正($V_\mathrm{out}$),伏秒积:
    \begin{equation}
        V_\mathrm{out}(1-D)T_s
    \end{equation}
    在稳态下,电感上的电流纹波$\Delta I$,应该等于MTOP开通期间输出电流的增加量,也等于其关断期间输出电流的减小量。
    所以开通和关断期间电感的伏秒积相等,
    \begin{equation}
        (V_\mathrm{in}-V_\mathrm{out})DT_s=V_\mathrm{out}(1-D)T_s
    \end{equation}
    可得:
    \begin{equation}
        D=\frac{V_\mathrm{out}}{V_\mathrm{in}}
    \end{equation}
    因此输出电流纹波:
    \begin{equation}
        \Delta I_L=\frac{\left(V_\mathrm{in}-V_\mathrm{out}\right)D}{Lf_s}=\frac{V_\mathrm{out}\left(1-\frac{V_\mathrm{out}}{V_\mathrm{in}}\right)}{Lf_s}
    \end{equation}
    \fbox{\makecell[{{p{16cm}}}]{由于输出电压的需求是确知的,开关频率由IC确定(为绕开汽车FM频段,一般固定在\SI{500}{\kilo\hertz}数量级或者大于2\si{\mega\hertz}),我们可以利用的结论通常总结为:输入电压越高,纹波越大;电感越大,纹波越小。}}

    有了电流纹波,就可以据此计算电感参数。一般原则是电流纹波为输出电流$I_\mathrm{out}$的$20\% \sim 40\%$。取小主要受到空间成本、动态响应等方面的约束,取大主要受到电容耐受、负载端PSRR指标的约束。


    个人观点:在负载为较高精度的模拟电路下,尽量选大电感加大耐压的电容,并做好负载IC的瞬态抑制和浪涌保护。


    电感值选取:
    \begin{equation}
        L=\frac{V_\mathrm{out}\left(1-\frac{V_\mathrm{out}}{V_\mathrm{in,max}}\right)}{\Delta I f_s}
    \end{equation}
    取输入电压最大值就是为了考虑最坏情况——如果$\Delta I$最大,电路能满足要求,那么其他情况纹波只会更小,更能满足要求。


    电感选取还要注意饱和电流的问题:铁氧体电感存在硬饱和、磁漏、体积大的问题,当电流超过一定值后电感感量急剧下降;一体成型的粉芯电感软饱和,但可能存在感值小、散热慢等问题。
    通常取
    \begin{equation}
        I_{L(max)}=I_\mathrm{out}+0.5 \Delta I
    \end{equation}
    作为对电感饱和电流参数的标准。必须选择饱和电流比该值大的电感。
    \begin{center}
        (饱和电流常定义为使L下降额定值的30\%处的电流)
    \end{center}


    通过电流纹波计算电压纹波:
    \begin{equation}
        V_\mathrm{ripple}=\Delta I\left(\frac{1}{8f_sC_\mathrm{out}}+(ESR)_{C_\mathrm{out}}\right)
    \end{equation}
    输出电压纹波有容性分量和阻性分量,分别由输出滤波电容的容值(不足)和其ESR(过大)引起。

    \subsection{PFC}
    功率因数校正(Power Factor Correction, PFC)电路,用于防止用电器产生的杂波影响到电网。
    
    LG=logic ground 逻辑地,也是数字地
    
    
    PE=power earth是电源地

    FG是保护地,外壳接地
