\chapter{概率论}

\begin{example}{}{几种概率分布的区别}
    令$x = \theta + w$ ,其中$w$是具有PDF $p_w(w)$的随机变量,如果$\theta$是一个确定性的参数,根据$p_w$求$x$的PDF,并且用$p(x;\theta)$表示。其次,假定$\theta$是一个与$w$独立的随机变量,求条件PDF $p(x|\theta)$。最后,不假定$\theta$和$w$是独立的,求$p(x|\theta)$。如何解释$p(x;\theta)$和$p(x|\theta)$?
    \tcbsubtitle{解:}
    
\end{example}